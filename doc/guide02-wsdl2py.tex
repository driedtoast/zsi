\chapter{\it{wsdl2py} basics}

The \WPY script is used to generate all the code needed to access a Web Service
through an exposed WSDL document, usually this description is available at a URL
which is provided to the script. 

\WPY will generate a client, types, and service module.   From the the WSDL
SOAP Bindings, the client and service modules are created.  The types module
contains typecodes for the schema defined by the WSDL.

\section{Modules}
\label{subsection: Modules}

\subsection{client stub module}

\subsubsection{classes}
The \it{service item} in the \WSDL definition contains one or more \it{port 
items}.

\paragraph{locator}
Defines a factory method for each \it{port item}, and stores the service's
address.  Use to grab a client(port) to the \WS.

\begin{verbatim}
# Example Locator
class WhiteMesaSoapRpcLitTestSvcLocator:
    SoapTestPortTypeRpc_address = "http://www.whitemesa.net/test-rpc-lit"
    def getSoapTestPortTypeRpcAddress(self):
        return WhiteMesaSoapRpcLitTestSvcLocator.SoapTestPortTypeRpc_address
    def getSoapTestPortTypeRpc(self, url=None, **kw):
        return Soap11TestRpcLitBindingSOAP(url or WhiteMesaSoapRpcLitTestSvcLocator.SoapTestPortTypeRpc_address, **kw)
}
\end{verbatim}

\paragraph{port}
Each \it{port item} will be represented by a single class definition, grab a
port through one of the locator's factory methods.

\begin{verbatim}
loc = WhiteMesaSoapRpcLitTestSvcLocator()
port = loc.getSoapTestPortTypeRpc(tracefile=sys.stdout)
\end{verbatim}


\paragraph{message} classes that represent the SOAP and XML Schema
data types. A \it{Message} instance is serialized as a XML instance.  A
\it{Message} passed as an argument to a \it{port} method is then serialized into
a SOAP Envelope and transported to the \WS, the client will then wait for an
expected response, and finally the SOAP response is marshalled back into the
\it{Message} returned to the user.

\begin{verbatim}
msg = echoBooleanRequest()
msg.InputBoolean = True
rsp = port.echoBoolean(msg)
\end{verbatim}

\subsection{types module}
Defines typecodes for all components of all schema specified by the target WSDL
Document (not including built-in types).  Each schema component declared at the
top-level, the immediate children of the \it{schema} tag, are global in scope
and by importing the "types" module an application has access to the GEDs and
global type definitions either directly or with the unique  
\it{(namespace,name)} combination thru convenience functions.

\subsubsection{classes}
\paragraph{Global Type Definition}
\begin{verbatim}
class ns1:
    ..
    ..
    class HelpRequest_Def(ZSI.TCcompound.ComplexType, TypeDefinition):
        schema = "http://webservices.amazon.com/AWSECommerceService/2006-11-14"
        type = (schema, "HelpRequest")
        def __init__(self, pname, ofwhat=(), attributes=None, extend=False, restrict=False, **kw):
            ..
\end{verbatim}
\par

\paragraph{Global Element Declaration}
\begin{verbatim}
class ns1:
    ..
    ..
    class Help_Dec(ZSI.TCcompound.ComplexType, ElementDeclaration):
        literal = "Help"
        schema = "http://webservices.amazon.com/AWSECommerceService/2006-11-14"
        def __init__(self, **kw):
            ..
\end{verbatim}
\par
\subsubsection{helper functions}
\paragraph{Global Type Definition}
\begin{verbatim}
    klass = ZSI.schema.GTD(\
        "http://webservices.amazon.com/AWSECommerceService/2006-11-14", 
        "HelpRequest")
    typecode = klass("Help")
\end{verbatim}
\par
\paragraph{Global Element Declaration}
\begin{verbatim}
    typecode = ZSI.schema.GED(\
        "http://webservices.amazon.com/AWSECommerceService/2006-11-14", 
        "Help")
\end{verbatim}
\par


Each module level class defintion represents a unique namespace, they're simply
wrappers of individual namespaces.  In the example above, the two inner classes
of \class{ns1} are the typecode representations of a global {\it type
definition} {\it \bfseries HelpRequest_Def}, and a global {\it element
declaration} {\it \bfseries Help_Dec}.  In most cases a \class{TypeCode}
instance represents either a global or local element declaration.

In the example \function{GED} returns a \class{Help_Dec} instance
while \function{GTD} returns the class definition \class{HelpRequest_Def}.  Why
this asymmetry?  
The element name is serialized as the XML tag name, while the type definition
describes the contents (children, text node).  

In the generated code an element declaration either defines all its content in
its contructor or it subclasses a global type definition, which is another
generated class.


\subsection{service module}
skeleton class, normally subclassed and invoked by implementation code.  The
skeleton defines a callback method for each operation defined in the SOAP
\it{Binding}.  These methods marshal/unmarshall XML into python types.

\subsubsection{example: DateService}
\paragraph{server skeleton code}
\begin{verbatim}
class simple_Date_Service(ServiceSOAPBinding):
    ..
    ..
    def soap_getCurrentDate(self, ps):
        self.request = ps.Parse(getCurrentDateRequest.typecode)
        return getCurrentDateResponse()

    soapAction['urn:DateService.wsdl#getCurrentDate'] = 'soap_getCurrentDate'
    root[(getCurrentDateRequest.typecode.nspname,getCurrentDateRequest.typecode.pname)] = 'soap_getCurrentDate'

\end{verbatim}
\paragraph{server implementation code}
\begin{verbatim}
DS = simple_Date_Service
class Service(DS):
    def soap_getCurrentDate(self, ps):
        response = DS.soap_getCurrentDate(self, ps)
        response.Today = today = response.new_today()
        self.request.Input
        dt = time.localtime(time.time())
        today.Year = dt[0]
        today.Month = dt[1]
        today.Day = dt[2]
        today.Hour = dt[3]
        today.Minute = dt[4]
        today.Second = dt[5]
        today.Weekday = dt[6]
        today.DayOfYear = dt[7]
        today.Dst = dt[8]
        return response
\end{verbatim}


\section{Generated TypeCodes}
The generated inner typecode classes come in two flavors, as mentioned above. 
{\it element declarations} can be serialized into XML, generally {\it type
definitions} cannot.\footnote{The \it{pname} can be set to \it{None} when a XML
tag name is not needed (eg. attributes).}   Basically, the {\it name}
attribute of an {\it element declaration} is serialized into an XML tag, but
{\it type definitions} lack this information so they cannot be directly
serialized into an XML instance.  

Most {\it element declaration}s declare a {\it type} attribute, this must
reference a {\it type definition}.  Considering the above scenario, a
generated {\it TypeCode} class representing an {\it element declaration} will
subclass the generated {\it TypeCode} class representing the {\it type 
definition}.

\subsection{special handling of instance attributes}
The attributes discussed below are common to all TypeCodes, for more information
see the ZSI manual.  I'm reintroducing them to point out certain conventions
adhered to in the generated code, necessary for reliably dealing with WSDL and
various messaging patterns and usages.

\subsubsection{pyclass}
All instances of generated {\it TypeCode} classes will have a {\it pyclass}
attribute, instances of the {\it pyclass} can be created to store the data
representing an {\it element declaration}.\footnote{Exceptions include the
Union TypeCode, may need multiple pyclasses to make it work}. The {\it pyclass}
itself has a {\it typecode} attribute, which is a reference to
the {\it TypeCode} instance describing the data, thus making {\it pyclass}
instances self-describing. 

When parsing an XML instance the data will be marshalled into an instance of the
class specified in the typecode's {\it pyclass} attribute.

\begin{verbatim}
    typecode = ZSI.schema.GED(\
        "http://webservices.amazon.com/AWSECommerceService/2006-11-14", 
        "Help")
    msg = typecode.pyclass()
\end{verbatim}
\par

\subsubsection{aname}
The {\it aname} is a {\it TypeCode} instance attribute, its value is a string
representing  the attribute name used to reference data representing an element
declartion. The set  of {\it XMLSchema} element names is {\it NCName}, this is
a superset of ordinary identifiers in {\it python}. Keywords like \it{return}
and \it{class} are legal NCNames.

\citetitle[http://www.w3.org/TR/REC-xml-names/]{Namespaces in XML}

\begin{verbatim}
From Namespaces in XML
	NCName	 ::=	(Letter | '_') (NCNameChar)*
	NCNameChar	 ::=	Letter | Digit | '.' | '-' | '_' | CombiningChar | Extender

From Python Reference Manual (2.3 Identifiers and keywords)
	identifier	::=	(letter|"_") (letter | digit | "_")*

Default set of anames
	ANAME	::=	("_") (letter | digit | "_")*
\end{verbatim}

\paragraph{transform} {\it NCName} into an {\it ANAME}
\begin{enumerate}
\item preprend "_"
\item character not in set (letter \verb!|! digit \verb!|! "_") change to "_"
\end{enumerate}
\par

\subsubsection{Attribute Declarations: attrs_aname}
The {\it attrs_aname} is a {\it TypeCode} instance attribute, its value is a string representing the
attribute name used to reference a dictionary, containing data representing 
attribute declarations.  The keys of this dictionary are the  
\verb!(namespace,name)! tuples, the value of each key represents the value of
the attribute.


\subsubsection{Mixed Text Content: mixed_aname}
Its value represents the attribute name used to store text content that some
\it{ComplexType} definitions allow.

\subsection{Metaclass Magic: pyclass_type}
\label{subsection:MM}
The {\it --complexType} flag provides many conveniences to the programmer. This 
option is tested and reliable, and highly recommended by the authors.

When {\it --complexType} is enabled the \verb!__metaclass__! attribute will be
set on all generated {\it pyclass}es.  The metaclass will introspect the {\it
typecode} attribute of {\it pyclass}, and create a set of helper methods for
each element and attribute declared in the {\it complexType} definition.  This
option simply adds wrappers for dealing with content, it doesn't modify the
generation scheme.

Use \it{help} in a python interpreter to view all the properties and methods of
these typecodes.  Looking at the generated code is not very helpful.

\begin{description}
\item[Getters/Setters] A getter and setter function is defined for each element
of a complex type.  The functions are named \verb!get_element_ANAME! and
\verb!set_element_ANAME! respectively.  In this example, variable \var{msg}
has functions named \verb!get_element__Options! and \verb!set_element__Options!.
 In addition to elements, getters and setters are generated for the attributes
 of a complex type.  For attributes, just the name of the attribute is used in
 determining the method names, so get_attribute_NAME and set_attribute_NAME are
 created.

\item[Factory Methods] If an element of a complex type is a complex type itself,
then a conveniece factory method is created to get an instance of that types
holder class.  The factory method is named, \verb!newANAME!.

\item[Properties]
\citetitle[http://www.python.org/download/releases/2.2/descrintro/#property]{Python class properties}
are created for each element of the complex type.  They are initialized with the
corresponding getter and setter for that element.  To avoid name collisions the
properties are named, \verb!PNAME!, where the first letter of the type's pname
attribute is capitalized.  In our running example, \var{msg} has class
property, \verb!Options!, which calls functions \verb!get_element__Options! and
\verb!set_element__Options! under the hood.

\end{description}

\subsubsection{example}
\paragraph{schema} 
Taken from the WolframSearch WSDL.

\begin{verbatim}
<xsd:complexType name='WolframSearchOptions'>
  <xsd:sequence>
    <xsd:element name='Query' minOccurs='0' maxOccurs='1' type='xsd:string'/>
    <xsd:element name='Limit' minOccurs='0' maxOccurs='1' type='xsd:int'/>
  </xsd:sequence>
  <xsd:attribute name='timeout' type='xsd:double' />
</xsd:complexType>
<xsd:element name='WolframSearch'>
  <xsd:complexType>
    <xsd:sequence>
      <xsd:element name='Options' minOccurs='0' maxOccurs='1' type='ns1:WolframSearchOptions'/>
    </xsd:sequence>
  </xsd:complexType>
</xsd:element>
\end{verbatim}
\par

\paragraph{help}(WolframSearchRequest)
\begin{verbatim}
Help on WolframSearch_Holder in module WolframSearchService_types object:

class WolframSearch_Holder(__builtin__.object)
 |  Methods defined here:
 |  
 |  __init__(self)
 |  
 |  get_element_Options(self)
 |  
 |  new_Options(self)
 |      returns a mutable type
 |  
 |  set_element_Options(self, value)
 |  
 |  ----------------------------------------------------------------------
 |  Properties defined here:
 |  
 |  Options
 |      property for element (None,Options), minOccurs="0" maxOccurs="1" nillable="False"
 |  
 |      <get> = get_element_Options(self)
 |  
 |      <set> = set_element_Options(self, value)
 |  
\end{verbatim}
\par

\paragraph{request}
\begin{verbatim}
from WolframSearchService_client import *
msg = WolframSearchRequest()
# get an instance of a Options holder class using factory method
msg.Options = opts = msg.new_Options()

# assign values using the properties or methods
opts.Query = 'Newton'
opts.set_element_Limit(10)

# don't forget the attribute
opts.set_attribute_timeout(1.0)

\end{verbatim}
\par

\paragraph{invoke}
\begin{verbatim}
port = WolframSearchServiceLocator().getWolframSearchmyPortType()
rsp = port.WolframSearch(msg)
print 'SearchTime:', rsp.Result.SearchTime
\end{verbatim}
\par

\paragraph{XML}
XML approximation of our WolframSearchRequest instance.
\begin{verbatim}
 <WolframSearch>
   <Options timeout="1.0" xsi:type="tns:WolframSearchOptions">
     <Query xsi:type="xsd:string">Newton</Query>
     <Limit xsi:type="xsd:double">10.0</Limit>
   </Options>
 </WolframSearch>
\end{verbatim}
\par






